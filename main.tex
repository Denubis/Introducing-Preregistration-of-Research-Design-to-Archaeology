\documentclass[article]{sa}
\PassOptionsToPackage{hyphens}{url}\usepackage{hyperref}
\usepackage{microtype}
\usepackage[utf8]{inputenc}

\title{Introducing Preregistration of Research Design to Archaeology}
\author{Shawn A. Ross\\Macquarie University \And Brian Ballsun-Stanton\\Macquarie University}
\date{For publication in 2022}

\Plainauthor{Shawn A. Ross, Brian Ballsun-Stanton}
\Plaintitle{Introducing Preregistration of Research Design to Archaeology}


\Abstract{
Archaeology has an issue with ``just-in-time'' research, where
insufficient attention is paid to articulating a research design before
fieldwork begins. Data collection, management, and analysis approaches
are under-planned and, often, evolve during fieldwork. While reducing
the amount of preparation time for busy researchers, these tendencies
reduce the reliability of research by exacerbating the effects of
cognitive biases and perverse professional incentives. They cost time
later through the accrual of technical debt. Worse, these practices
hinder research transparency and scalability by undermining the quality,
consistency, and compatibility of data. Archaeologists would benefit
from embracing the ``preregistration revolution'' sweeping other
disciplines. By publicly committing to research design and methodology
ahead of time, researchers can produce more robust research, generate
useful and reusable datasets, and reduce the time spent correcting
problems with data. Preregistration can accommodate the diversity of
archaeological research, including quantitative and qualitative
approaches, hypothesis-testing and hypothesis-generating research
paradigms, and place-specific and generalizing aims. It is appropriate
regardless of the technical approach to data collection and analysis.
More broadly, it encourages a more considered, thoughtful approach to
research design. Preregistration templates for the social sciences can
be adopted for use by archaeologists.}


\Keywords{
archaeology, preregistration, open data, research design, research transparency}
\Plainkeywords{
archaeology, preregistration, open data, research design, research transparency}

\Journal{Preprint}
\Volume{}
\Issue{} %Enter as "(Issue)" for APA
\Pages{} %Enter as "pp. m---n" for APA
\Year{}
\Submitdate{22 April 2021} %Enter date you submitted preprint for review
\Acceptdate{2 April 2020} %Enter date your paper was accepted for publication
\DOI{10.31235/osf.io/sbwcq} %Enter final DOI here

\Address{
    Shawn A. Ross\\
    Department of History and Archaeology, Faculty of Arts, Macquarie University\\
    Sydney, Australia\\
    E-mail: \email{shawn.ross@mq.edu.au}\\
    URL: \url{http://orcid.org/0000-0002-6492-9025}\\
    \\
    Brian Ballsun Stanton\\
    Faculty of Arts, Macquarie University\\
    Sydney, Australia\\
    E-mail: \email{brian.ballsun-stanton@mq.edu.au}\\
    URL: \url{https://orcid.org/0000-0003-4932-7912}

}

\begin{document}



\subsection*{Cite this Preprint}

{\small
\hangindent=0.5cm{}Ross, Shawn A., and Brian Ballsun-Stanton. Accepted 2 April 2020. ``Introducing Preregistration of Research Design to Archaeology.'' In \textit{Digital Heritage and Archaeology in Practice}, edited by Ethan Watrall and Lynne Goldstein. Gainesville, FL: University Press of Florida.

 }


\newpage

Preregistration of research design, and in some cases hypotheses, has
been promoted as a technique to combat researchers' biases, overcome
perverse professional incentives, improve transparency, and increase
rigor. The motivation stems from the ``reproducibility crisis'' faced by
many disciplines (Nature 2016; Open Science Collaboration 2015; Adam
2019; Kaplan and Irvin 2015; Franco et al. 2014; Mellor et al. 2016;
Baker 2015; Munafò et al. 2017; National Academies of Sciences,
Engineering, and Medicine et al. 2019). Preregistration can be described
as the declaration of a research plan before data collection begins
(Center for Open Science 2018), or a process in which researchers:
``define the research questions and analysis plan before observing the
research outcomes'' (Nosek et al. 2018a). Although often associated with
scientific research, many disciplines can benefit from preregistration.
Preregistration accommodates inductive (hypothesis-generating;
postdictive) as well as deductive (hypothesis testing; predictive)
research, idiographic (distinct to a particular place) as well as
nomothetic (applicable to a larger class of events of conditions)
approaches, qualitative as well as quantitative analyses, and can be
applied at various stages of the research lifecycle (Nosek et al.
2018a). In archaeology and other field sciences, preregistration offers
its usual benefits: explicit recognition of research design; management
of biases and perverse incentives; and exposure of research methods and
processes as required by emerging good practice in research transparency
(Center for Open Science 2019).

Preregistration could also counteract a sociotechnical problem hindering
the adoption of digital field methods: a reluctance to invest time and
resources in the early planning and preparation phases of a project,
versus time later during fieldwork, post-processing, and analysis
(Sobotkova et al. 2016). This lack of planning leads to changes in
research design during execution with little accountability, including
for example, ongoing modification of data structures and data capture
workflows during fieldwork (Borgman 2015). Such in-progress changes
raise the cost and complexity of digital field data collection systems.
\emph{Ad hoc} changes limit data quality by precluding adequate testing
and refinement, undermining consistency of data and methods, and
imposing ``technical debt'' such as extensive cleaning and reconciliation
of data at the end of the project. These ``easy'' changes during and after
fieldwork thereby hinder the interoperability and reusability of
resulting datasets.

This chapter will explore how preregistration can benefit domains like
archaeology that involve approaches spanning the humanities and the
sciences, with both deductive and inductive elements. We discuss the
benefits of preregistration for archaeological research, argue that
preregistration is feasible for archaeologists, and propose how
preregistration might be implemented based on a model from qualitative
research in the social sciences.

\section{Preregistration Accommodates
Predictive and Postdictive Approaches}

Preregistration has been promoted as a way to improve confidence in the
results of research, especially by clarifying whether research generates
new hypotheses from existing data, or tests existing hypotheses using
new data. Maintaining this distinction combats researchers' cognitive
biases, ensures that research adheres to assumptions intrinsic to common
statistical approaches, and counteracts the distortions introduced by
perverse professional incentives that encourage questionable research
practices. For example, we can see a ``publication bias'' towards papers
that confirm, rather than refute, initial hypotheses in deductive
research (Chase 2013; Fraser et al. 2018). Although preregistration is
often associated with scientific and biomedical research
(\url{https://clinicaltrials.gov/} is the largest existing registry), it
is also used in social sciences like economics and political science
(see \url{https://www.socialscienceregistry.org/};
\url{http://egap.org/content/registration/}; cf. Nosek et al. 2018a; Haven and
Van Grootel 2019).

Preregistration emerged as a new methodological commitment in
disciplines like psychology and biomedicine that have struggled to
replicate or reproduce results. The problems it addresses, however,
exist in many domains. Nosek, for example, emphasizes researchers'
tendency to conflate ``prediction'' and ``postdiction''. Prediction
describes a deductive, hypothesis-testing approach since the hypothesis
predicts outcomes of empirical research. Postdiction describes an
inductive, hypothesis-generating approach since the hypothesis arises
from (and after) empirical research (Nosek et al. 2018a; Haven and Van
Grootel 2019). Predictive research provides a rigorous way to test and
disprove hypotheses. Such testing is a boon for positivists, who remind
us that Popper's injunction that ``proof'' of a hypothesis is always
provisional while disproof is final, making falsifiability the hallmark
of science (Derksen 2019). Postdictive research, conversely, can
generate surprising or unexpected hypotheses that radically shift our
understanding of the world or even change scientific worldviews (Kuhn
1970:23).

Both predictive and postdictive approaches are essential, but they must
not be conflated. Therefore, even if a researcher does not hold with
positivist Popperian falsification as the primary mechanism for
scientific discovery (Huemer 2020), formalisms like preregistration that
enforce distinctions between prediction and postdiction improve the
quality of research:

\begin{quote}

``Failing to appreciate the difference can lead to overconfidence in
\emph{post hoc} explanations (postdictions) and inflate the likelihood
of believing that there is evidence for a finding when there is not.
Presenting postdictions as predictions can increase the attractiveness
and publishability of findings by falsely reducing uncertainty'' (Nosek
et al. 2018a).

\end{quote}

If postdiction is conflated with prediction, it is prone to ``fallibility
of memory, motivated reasoning, and cognitive biases'' (Nosek et al.
2018a). Of particular concern is hindsight bias, where researchers
(often subconsciously) remember or seek evidence supporting their
conclusions while ignoring or faulting contrary evidence. Since common
statistical methods like null hypothesis significance testing, moreover,
assume prediction rather than postdiction, they are likely to produce
``Type 1 errors'' (false positives) when researchers neglect the
difference. As a result of these problems, ``Mistaking postdiction as
prediction underestimates the uncertainty of outcomes and can produce
psychological overconfidence in the resulting findings'' (Nosek et al.
2018a). Although postdictive research can produce new inconvenient facts
that contradict received paradigms, it should not simultaneously test
the hypotheses it generates.

Indeed, ``using current results to construct \emph{post hoc} hypotheses
that are then reported as if they were \emph{a priori} hypotheses'',
``failing to report \emph{a priori} hypotheses that are unsupported by
the current results'' (Rubin
2017), or ``presenting exploratory work as though it was confirmatory
hypothesis testing'' (Fraser
et al. 2018), is considered ``hypothesizing after the results are known''
or HARKing (Kerr 1998). HARKing is always a questionable research
practice if it is unreported, although disagreement exists about the
acceptability (or even desirability) of careful and transparent
\emph{post hoc} analysis in deductive research (Rubin 2017; Hollenbeck
and Wright 2017). In a 2018 paper surveying over eight hundred
ecologists and evolutionary biologists, 51\% admitted to HARKing (Fraser
et al. 2018). Fraser also notes in passing that when published papers
fail to disclose \emph{a priori} hypotheses (or if there were \emph{a
priori} hypotheses), it becomes difficult to judge whether HARKing --
or the conflation of postdiction and prediction more broadly -- has
even taken place. As such, the \emph{a priori} articulation of
hypotheses required by preregistration (or the explicit statement that
research is inductive) can help to combat this species of questionable
research practice.

\section{The Problem of ``Just-in-Time''
Archaeology}

The biases and perverse incentives that preregistration was designed to
combat in other disciplines also exist in archaeology. Processual
archaeology's ``scientific approach'', with its emphasis on revealing aims
and biases and its problem-orientation, has recognized these problems
for half a century (e.g. Johnson 2010:24--25). Indeed, the relevance of
early processual thought to the current problem is striking. In the
1970s, Hole criticized archaeologists' tendency to focus on fieldwork
and analytical methods rather than research design, especially the
development of hypotheses or models. He called for more ``thought'' before
the archaeologist turned to ``methodological manipulations'' (Hole
1973:32). In the same proceedings, French observed that:

\begin{quote}
    
``{[}I{]}t is essential and (?) obligatory to define the problems before
developing or choosing the means used to collect the data necessary to
examine these problems and the hypotheses explicit in them. In other
words, the questions asked dictate the methods of recovery; one selects
the recovery technique to suit the problem in hand'' (French 1973:105).

\end{quote}

French went on to emphasize the importance of what would now be termed
data provenance. Data provenance documents where material was found or
observations made, by what method, under what conditions, and with what
sampling approach. He further recommended that archaeologists specify
the program and standards governing data collection. Only by doing so
can the quantity, quality, and comparability of archaeological data be
assessed. The situation French described -- a lack of shared (or at
least articulated) standards of data acquisition and documentation --
remains widespread today, undermining the comparability of data from
different projects (French 1973:106; Atici et al. 2013; Faniel et al.
2013; Holub et al. 2018). Influential recoomendations, such as those
encapsulated by Findable, Accessible, Interoperable, and Reusable (FAIR)
data and the Transparency and Openness Promotion (TOP) Guidelines
likewise emphasize the creation of proper metadata -- the
documentation of data such that it can be understood and reused by
outsiders (Wilkinson et al. 2016; Center for Open Science 2019). In
short, merely collecting more data does not automatically support the
syntheses needed to produce a better understanding of the past, and
data-recovery techniques must harmonize with theoretical approaches
(e.g. Johnson 2010:24--25; French 1973:107). Preregistration encourages
the planning and preparation necessary to collect data appropriate to a
project's aims now and useful to others in the future.

Preregistration can, moreover, help researchers articulate their
specific theoretical approach and research tradition. Processualism and
post-processualism both have fundamental assumptions about the nature of
evidence, approaches to knowledge, and the interpretation of data. They
are examples of competing Lakatosian research programs, each with their
own ``hard-core'' of truths which they take to be self-evident and ``soft
outer shell'' of current, unproven, research questions (Lakatos 1978).
Preregistration makes such researcher assumptions more explicit (Johnson
2010:3).

Preregistration also mitigates a specific sociotechnical barrier to the
adoption of digital approaches in archaeology. Archaeologists tend to
substitute remedial work at the end of a project for proper planning and
preparation at the beginning (Sobotkova et al. 2016). This tendency,
along with the continuing prevalence of print publication of datasets
like artifact catalogs, impedes the production of comprehensive,
reusable datasets in archaeology. It also undermines the cultivation of
deeper digital practice more generally.

This sociotechnical barrier to innovation in the discipline warrants
further discussion. Our information infrastructure work, related to the
Field Acquired Information Management Systems (FAIMS) project, has
customized field data collection systems for over sixty workflows at
more than forty projects in archaeology, ecology, geoscience, history,
and other disciplines since 2014 (Ross et al. 2013, 2015; Sobotkova et
al. 2015, 2016; Ballsun-Stanton et al. 2018; VanValkenburgh et al.
2018). We have found that field researchers broadly, and archaeologists
particularly, under-invest in planning and preparation in three ways.

First, archaeologists tolerate a lack of detail or precision in their
documentation. For example, they do not take the time to build and
deploy controlled vocabularies or they leave important and recurring
information for ``notes'' or other free-text fields. When applied to
paper-based recording, these practices exploit the forgiving nature of
the medium, relying on the ability to write ``in the margins'' or ``on the
back of the page''. In digital recording, these tendencies often
translate into poorly designed and unvalidated spreadsheets or
databases. This haste and imprecision reduces the up-front investment in
data and workflow modeling for harried academics and consultants but
adds an implicit cost: a much larger burden of post-fieldwork data
cleaning.

Second, forms and protocols assume a great deal of implicit knowledge
transmitted orally, often informally, at the project. This practice
reduces the transparency of fieldwork and fails to record important
metadata needed later for data reuse. Implicit methods and metadata are
lost by team members, volunteers, and students who forget just how
things were done, are uncontactable, or who have moved onto other
projects. This epistemic turnover impedes later understanding and reuse
by outsiders -- or even project participants -- who cannot retrieve lost
context.

Third, project leaders tend to develop record-keeping or documentation
practices like forms and protocols late, shortly before fieldwork
begins, leaving insufficient time to test and refine them with realistic
trial data. This practice exacerbates the effects of the first two
problems, producing records that only approximate the documentation that
researchers desire. It also contributes to a feature common to many
fieldwork-based domains, a tendency for research design to be ``emergent''
-- for researchers to elaborate and modify approaches, methods, tools,
technologies, and practices during fieldwork when working with real data
(Borgman 2015:106--107). Although a certain amount of ``tinkering''
(c.f. Sobotkova and Janouchova 2021, Chapter 6 this volume) may be
necessary to mitigate the unpredictability of fieldwork, the practice
produces datasets that are not even consistent or comparable across a
single season at a single project without considerable manual
reconciliation, let alone more broadly.

Together, these practices constitute a sociotechnical hurdle to
effective digital archaeology: underinvestment in the early stages of a
project and subsequent accrual of technical debt, requiring the
expenditure of much more time and effort to ``fix'' data at the end of a
project (Sobotkova et al. 2016). This misallocation of time extends to
-- and is exacerbated by -- an under-allocation of research funds to
data management (Mons 2020). In data quality and reusability, as much as
in analytical reproducibility, such ``bolt-on'' fixes applied at the end
of a project are much more costly than ``built-in'' good
practice, if they are feasible at all (Marwick 2017b). Perhaps of
greatest concern, late data cleaning is often ``lossy'', irreversibly
reducing the quality and utility of a project's data, a serious
consideration due to the unrepeatable nature of most archaeological
research.

Addressing this sociotechnical problem by improving research design,
data production, and the relationship between them is crucial. The
necessity of producing quality, consistent, reusable data is now
recognized in archaeology and beyond. Data is becoming a first-class
output and a key element of academic rigor. Publication of analytical
code is likely to follow. Pressure is growing to improve the
transparency of research by, for example, making data and other digital
research objects FAIR (Wilkinson et al. 2016; GO-FAIR 2017; Stall et al.
2019). That pressure includes requirements established by journals,
funders, and regulators, as well as broader pressure to implement better
practice in research. For example, sharing of FAIR data has been written
into the (Australian) National Statement on Ethical Conduct in Human
Research and the Australian Research Council's Open Access Policy, both
promulgated by the National Health and Medical Research Council (2018,
2019). Publishers representing over a thousand journals and many private
funders have endorsed the TOP Guidelines (Mellor et al. 2016; Center for
Open Science 2019). Meeting such requirements will require a greater
investment, both in time and resources, from researchers in early
planning and preparation. In the future, stakeholders will expect
archaeologists' data and code, not just our results and conclusions. At
present, however, data repositories are underpopulated and datasets are
not routinely reused (McNutt et al. 2016; Sobotkova 2018). This
underpopulation inhibits reproduction or verification of results,
independent analyses of primary data, and the application of new
techniques to old datasets. It also prevents the combination of datasets
from multiple studies for large-scale research to address ``grand
challenges'' (Snow et al. 2006; Kintigh 2006; Kintigh et al. 2014; McNutt
et al. 2016).

\section{Preregistration is Suitable for
Archaeology}

Preregistration can help resist the temptation of ``just-in-time''
research design. It would make ``built-in'' good practice more likely, as
opposed to more common attempts to ``bolt-on'' fixes after fieldwork
(Marwick 2017b:441). Specifically, preregistration offers a mechanism to
make approaches and assumptions explicit, allowing them to be
scrutinized (Johnson 2010:3) and counteracting cognitive biases.
Preregistration also helps researchers overcome the sociotechnical
tendency to underinvest in preparation for data acquisition and
analysis, instead of relying on remediation of problems late in the
research lifecycle. Combined with a commitment to other aspects of
reproducibility (Perkel 2018; Marwick 2017b; Wilkinson et al. 2016),
preregistration holds the potential to improve research practice
tangibly in our discipline.

Preregistration can, moreover, accommodate the diversity and
transdisciplinarity of archaeological research. Archaeology can be
deductive and predictive or inductive and postdictive. Most often,
archaeology -- like other fieldwork disciplines -- might best be
described as abductive (Tavory and Timmermans 2014). Abductive research
represents a synthesis of deductive and inductive approaches, gathering
data according to a specific and intentional methodology (the inductive
element), and then applying a theory or framework that describes those
facts (deductive). Used in interpretive social science research, this
methodology is seldom explicitly acknowledged or described (Lewis-Beck
et al. 2003:1). Archaeology can also be quantitative and statistical or
qualitative and descriptive. It can be idiographic, seek an in-depth
understanding of the specific, contingent, and unique in a particular
place and time. Or, it can be nomothetic, attempting to derive or test
general (if provisional or incomplete) principles or rules by looking at
similarities across space and time. Idiographic research is often
associated with inductive and qualitative approaches in the humanities.
Nomothetic research is usually associated with deductive and
quantitative approaches in the social and natural sciences. These
approaches, however, can cut across disciplines and can be combined,
iterated, or synthesized in various ways (Watson 1973:51). A
preregistration regime does not privilege any of these approaches over
another but instead demands an articulation of, and a public commitment
to, a specific research design at the beginning of a project,
\emph{before} research begins. Indeed, the very complexity and diversity
of archaeology -- spanning disciplines and approaches to knowledge
-- makes articulation of research design particularly important. An
explicit research design documented in detail before research starts,
rather than during the writing-up stage, avoids the unintentional
conflation or elision of approaches that could undermine research
results.

Archaeology is also serendipitous. Often, we do not know what we will
find until we get to the field. This problem, however, is not unique to
archaeology. Even in deductive disciplines, results are not and should
not be under the control of the researcher
(Chambers 2019). Research
design, research questions or hypotheses, methods, and interpretive
frameworks, however, are under researcher control. As a result, these
elements can be articulated beforehand. Training, prior fieldwork,
knowledge of analogous sites or landscapes, regulator or funder
requirements, and disciplinary expectations around fieldwork and
publication all inform research. These prior requirements and
similarities allow archaeologists to state \emph{why} and \emph{how}
they plan to undertake research, if not \emph{what} they will find.
Preregistration forces a level of clarity and formalizes an \emph{a
priori} commitment to research design and recording strategy. Indeed,
since expected outcomes of research vary, and are often externally
dictated (e.g., by heritage legislation), the articulation of research
design is a particular benefit to archaeologists since it moves
preparation to the less-fraught time before fieldwork begins. This
preparatory work allows thoughtful development, testing, and revision of
approaches, increasing the chances that the research design and
collected data will support the desired outcomes. Finally, we recognize
that exigencies of fieldwork may suggest a ``pivot'' due to an
extraordinary find or the late discovery of unsuitability in approach or
documentation during execution. Preparing for the unexpected again
argues for careful \emph{a priori} research design (e.g., consideration
of abductive elements within deductive research), while the planning
encouraged by preregistration should help to avoid errors in design
(especially if combined with pilot research). Considering its potential
value, the unpredictable nature of fieldwork should not preclude a
``best-effort'' at preregistration -- and a ``best-effort'' in an
abductive discipline like archaeology should be considered sufficient.

\section{Introducing Preregistration to
Archaeological Practice}

Despite its potential utility, we could find no examples of rigorous
preregistration in archaeology. A search of OSF's 304,904 registrations
(as of March 19, 2020) produced only four non-teaching-related hits for
``archaeology''
(\href{https://osf.io/registries/discover?provider=OSF\&q=archaeology}{{https://osf.io/registries/discover?provider=OSF\&q=archaeology}}).
The OSF differentiates between categories of registration. The authors
of the four preregistrations, Marwick (2017a, 2017c), Selden (2016), and
Schmid (2019) used the ``Open-Ended Registration'' method to submit data
and code in support of papers they had written. While the submission of
code in support of scientific papers is laudable, these FAIR data
``registrations'', which support replication of computational results, do
not represent the preregistration of approaches and methods we suggest.

Selden's pottery laser-scanning project (2017), for example, would be a
good candidate for the preregistration of an inductive, postdictive
project. A more complete preregistration effort, in this case, could
include a protocol for the laser-scanning methodology and, crucially, a
protocol for the creation and publication of metadata. Spot checks
indicate that data documentation is incomplete; metadata like intra-site
find location and artifact type do not accompany some scans (e. g.
\href{https://zenodo.org/search?page=1\&size=20\&q=41NA49}{{https://zenodo.org/search?q=41NA49}}).

Likewise, Marwick et al.'s lithic trampling experiment (2017a)
represents an opportunity for the preregistration of more deductive,
predictive research. This study explored, ``claims of vertical movement
of artifacts in debates surrounding the timing of the first human
occupation,'' creating a physical experiment on the soil heaps of
Madjedbebe (Marwick 2017a). Preregistration could articulate hypotheses,
methodology, and an analytical approach. Perhaps, this preregistration
could even include early versions of code. With moderate effort,
especially if a grant application exists for this research, it could be
elaborated to a full preregistration plan, or even a ``registered
report'', in which a journal accepts or rejects the paper on the strength
of its hypothesis, methodology, and proposed analysis before results are
known (see Chambers 2019).

We note that Selden's presentation of the laser scan data with licensing
information exceeds standard practice in archaeology, complements the
associated paper (2017), and should be commended. Likewise, Marwick's
publication of containerized code is exceptional, confirming his status
as a leader in computational reproducibility. Rather, these examples
highlight that (1) even archaeologists committed to transparency and
reproducibility have not adopted comprehensive preregistration of
approach, method, and data management, and (2) undertaking
preregistration would reinforce other efforts to improve transparency
and reproducibility, like FAIR datasets and publication of analytical
code.

Templates, examples, and models could facilitate the introduction of
preregistration into archaeology. Although a protocol for archaeological
preregistration is beyond the scope of this chapter, and good practice
around the execution of preregistration is still evolving (c.f. Nosek et
al. 2018a; Ledgerwood 2018; Nosek et al. 2018b), we can offer some
guidelines based on practices in other disciplines.

A distinction is sometimes made between preregistration of research
questions or hypotheses on the one hand, and plans for analysis on the
other:

\begin{quote}
    
``{[}P{]}reregistration of theoretical predictions helps researchers know
how to correctly calibrate their confidence that a study tests (versus
informs) a theory, whereas preregistration of analysis plans helps
researchers know how to correctly calibrate their confidence that a
specific finding is unlikely to be due to chance.'' (Ledgerwood 2018)

\end{quote}

In the case of archaeology, experience with the FAIMS Project indicates
that two additional aspects of archaeological research might also be
registered profitably: data models and data workflows including data
capture, manipulation, and analysis. Recording these aspects of research
will help archaeologists meet emerging standards such as the TOP Level 2
Guidelines, which require ``a full account of the procedures used to
collect, preprocess, clean, or generate the data'' (data provenance) and
a ``description of procedures'' necessary for an ``independent replication
of the research'' (Center for Open Science 2019). Thereby acknowledging
that the abductive and serendipitous nature of the discipline may often
make transparency the goal of preregistration, rather than replication.
As noted above, the perfect should not be made the enemy of the good,
and a best-effort attempt at preregistration would mark an improvement
over the present lack of any at all.

Preregistration in archaeology could therefore include declarations of
research tradition (e.g., theoretical framework), approach (e.g.,
inductive, deductive, idiographic, nomothetic, etc.), hypotheses or
research questions, fieldwork and analysis plans, workflows, and/or data
models. Compared to the current state of affairs, articulation of any
aspect of research design would mark an improvement. An explicit
commitment to a particular approach plus either hypotheses or research
questions would do the most to combat researcher biases. Articulation of
data models and workflows would do the most to overcome ``just-in-time''
fieldwork and produce reusable, widely comparable data at the end of the
project.

In choosing a preregistration protocol, archaeologists will face a
trade-off between the amount of time spent defining their approach and
the resulting loss of flexibility. A ``stricter'' preregistration will
maximize the benefits of preregistration, whereas a ``looser''
preregistration is faster and less constraining -- but also less
effective at avoiding bias or ``just-in-time'' fieldwork.

A preregistration regime in archaeology could be built using existing
knowledge infrastructures, like the Open Science Framework (Open Science
Framework 2020; Bowman 2019). OSF provides a range of preregistration
templates that could be adapted for archaeological use (Mellor and
DeHaven 2016), including templates for ``qualitative research'' that
accommodate the nature of archaeology as an abductive discipline while
``providing a check on subjectivity'' (Haven and Van Grootel 2019).

Archaeology would benefit from the development of domain-specific
registration templates, such as that for Social Psychology (van 't Veer
and Giner-Sorolla 2016), or perhaps for specific types of research, on
the model of the ``replication recipe'' for conducting replication
experiments in various disciplines (Brandt et al. 2014). We hope that
the current chapter can inform the development of such templates,
perhaps in consultation with the proprietors of archaeological data
repositories like Open Context (see Kansa and Kansa 2010; Kansa and
Bissell 2010), tDAR (McManamon and Kintigh 2016; McManamon et al. 2017),
and the Archaeology Data Service, whose data dissemination and reuse
expertise complements the field data management experience of the FAIMS
Project.

Until archaeology-specific templates and protocols exist, we recommend
starting with the OSF ``Qualitative Research Preregistration'' template
(\href{https://osf.io/6z2hr/}{{https://osf.io/6z2hr/}}; see Haven and
Van Grootel 2019). It is intended for a range of social science
research;not all fields are relevant to archaeology, and some
archaeology-specific fields will probably need to be added. In our view,
key fields from this template that are relevant to archaeology include:

\begin{itemize}
\item  All ``Study information'' items (points 1-5), which provide basic
  project metadata and commitment to particular research questions
  should be included. ``Typical moments'' of research-question
  modification represents a useful concept that accommodates the
  emergent and serendipitous nature of archaeological research. As noted
  at 5.1, inductive vs. deductive should be indicated, and we would
  suggest adding idiographic vs. nomothetic. Projects with more of a
  deductive element should include any hypotheses that will be tested
  (5.2). Further information, such as the extent and nature of
  qualitative vs. quantitative analysis might also be useful. The ``Use
  of Theory'' section could likely be simplified and made more
  archaeology-specific.
\item   ``Tradition'' (Point 6) might be combined into the prior ``Use of Theory''
  section, especially 5.3. This condensed section could include, the
  specific paradigm of fact-generation: ``processualist,''
  ``post-processualist,'' ``structuralist,'' ``marxist,'' ``cognitive
  archaeology,'' etc. (examples from Johnson 2010), The articulation of
  the paradigm used to generate facts may serve to contextualize the
  details of design, sampling, and analysis below.
 \item  Regarding ``Design Plan'' (points 7-8), ``Study Type'' should be
  specified as ``excavation,'' ``surface survey'', ``artifact
  analysis'', ``geoarchaeology'', etc., while ``Study Design'' is
  required for most grant or permit applications and could be generated
  from those sources. It would be useful to develop an
  archaeology-specific enumerated list of study types, and perhaps also
  of design elements linked to common approaches, rather than relying
  exclusively on a narrative study design.
 
\item   Under ``Sampling Plan'' (points 9-17), existing vs. non-existing data
  (and ``Explanation of Existing Data'' if relevant), ``Data Collection
  Procedures'' (adjusted for archaeology), ``Type of Data Collected'',
  ``Sort of Sample'', and ``Data Collection Plan'' should be specified.
  ``Type of Sampling Rational'' may be useful for quantitative or
  geospatial research. These categories might be modified or extended to
  ensure that the data model and data capture workflow are defined.

\item   ``Script'' (point 18) should read ``Forms'' (whether paper or digital)
  since it indicates ``Data Collection instruments''. Note again that
  modification during research is accommodated.

\item  An ``Analysis Plan'' (point 19) should be articulated, but with the
  categories at 19.1 amended to reflect archaeological practice.
  Important information might include the balance between quantitative
  and qualitative analysis, and/or a statement of the approach used for
  either type (e.g., the statistical approaches, application
  middle-range theory, etc.). It might also be useful to break the
  analytical procedures or workflow out from the descriptive section
  (19.2.1) into its own enumerated list.
  
\item ``Other'' (point 20) provides a space for additional information that
  will help others to understand and reuse your research.
 
\end{itemize}

Beginning with a broad, existing template such as this one allows
experimentation that, over time, can produce preregistration protocols
that reflect and respect archaeological practice.

\section{Preregistration as ``Slow Archaeology''}

The discipline of oceanography has also grappled with ``small data''
problems (Borgman 2015; Kansa and Bissell 2010), including diverse data
and data structures that emerge from fieldwork, producing internally
inconsistent datasets and mid-course changes to research. These problems
include those that shade into questionable research practices, judging
from Fraser et al.'s (Fraser et al. 2018) survey of related disciplines.
As in archaeology, these problems inhibited the large-scale, synthetic
research needed to address grand challenges in oceanography. In the
course of aggregating and synthesizing data from many sources, the Ocean
Health Index project discovered that:

\begin{quote}

``Environmental scientists are expected to work effectively with
ever-increasing quantities of highly heterogeneous data even though they
are seldom formally trained to do so... Without training, scientists
tend to develop their own bespoke workarounds to keep pace, but with
this comes wasted time struggling to create their own conventions for
managing, wrangling and versioning data. If done haphazardly or without
a clear protocol, these efforts are likely to result in work that is not
reproducible - by the scientist's own `future self' or by anyone else.''
(Lowndes et al. 2017).

\end{quote}

The Ocean Health Index project's response to scalability and
reproducibility was to implement data collection, manipulation, and
analysis techniques based on open data formats, data standardization,
and script-based analysis. Then, they created libraries of scripts that
could perform commonly required analyses on well-structured data\emph{.}
The resulting data and associated analytical scripts could be shared,
aggregated, and reused by researchers around the world. Increases in the
quantity of available data and improving data quality were functions of
reducing friction, avoiding technical debt, and implementing
requirements around data formats, data structures, and analytical
methods. More important than the generation of standardized, reusable
data and scripts, however, was the thoughtful and planned research
design, including its approach to creating, recording, and analyzing
data.

Such careful planning not only aligns with the early processualists'
call for more emphasis on thoughtful, \emph{a priori} research design,
but also with Caraher's recent argument for a ``slow archaeology'':

\begin{quote}
    
``Slow archaeology evokes the practice of archaeology as a craft. It
prioritizes an embodied attentiveness to the entire process of fieldwork
as a challenge to the fragmented perspectives offered by workflows
influenced by our own efficient, industrialized age'' (2015).

\end{quote}

Preregistration and associated digital approaches that support
transparency and reproducibility are not antithetical to slow
archaeology -- the chaos of ``just-in-time'' approaches to fieldwork
are. Under-planned fieldwork might offer more freedom (and delay the day
of reckoning with messy data), but it represents the opposite of the
considered approach suggested by Caraher and is likely to result in
biased and opaque research that does not reach its full potential.
Preregistration encourages thoughtfulness.

\section{Conclusions}

Calls for archaeologists to improve data quality for large-scale
research have been made for at least the past two decades (Doerr et al.
2004; Kansa and Kansa 2011; Faniel et al. 2013; Austin 2014), but
limited progress has been made (Kintigh et al. 2014; Sobotkova 2018).
Concerns over the transparency and reproducibility of archaeological
research are more recent (Marwick 2017b), but archaeology is unlikely to
avoid the reproducibility crisis so profoundly affecting other
disciplines. Our recommendation to adopt preregistration as a means of
increasing rigor is a pathway to solving those challenges.
Preregistration promotes ``built-in'' rather than ``bolt-on'' good practice
in research design, data management, and analysis. It likewise fosters
thoughtful, ``slow'' archaeology rather than ``just-in-time'' archaeology.
Robust, scalable, transparent, and reproducible results, underlying
persuasive research directed at grand challenges, require planning and
forethought. Making a public commitment to research design \emph{via}
preregistration, before setting out for the field, provides a mechanism
that makes space for the necessary time and thought.

%\section{References}
\nocite{*}
\bibliography{references}

\end{document}
